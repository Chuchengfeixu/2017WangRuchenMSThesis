\begin{resume}

  \resumeitem{个人简历}

  1992 年 5 月 14 日出生于 山东 省 威海 市。

  2010 年 9 月考入 浙江师范 大学 数理与信息工程 学院 电子信息工程 专业,2014 年 7 月本科毕业并获得 学士 学位。

  2014 年 9 月考入 中国海洋 大学 电子 系攻读 硕士 学位至今。

  \researchitem{发表的学术论文} % 发表的和录用的合在一起

  % 1. 已经刊载的学术论文(本人是第一作者,或者导师为第一作者本人是第二作者)
  \begin{publications}
    \item Wang R, Dai J, Zheng H, Ji G, Qiao X. Multi features combination for automated zooplankton classification, in Proceedings of OCEANS’16 MTS/IEEE Shanghai, 2016.
    \item Gu Z, Wang R, Dai J, Zheng H, Zheng B. Automatic searching of fish from underwater images via shape matching, in Proceedings of OCEANS’16 MTS/IEEE Shanghai. 2016.
  \end{publications}

  % 2. 尚未刊载,但已经接到正式录用函的学术论文(本人为第一作者,或者
  %    导师为第一作者本人是第二作者)。
  % \begin{publications}[before=\publicationskip,after=\publicationskip]
  %   \item Yang Y, Ren T L, Zhu Y P, et al. PMUTs for handwriting recognition. In
  %     press. (已被 Integrated Ferroelectrics 录用. SCI 源刊.)
  % \end{publications}

  % 3. 其他学术论文。可列出除上述两种情况以外的其他学术论文,但必须是
  %    已经刊载或者收到正式录用函的论文。
  % \begin{publications}
  %   \item Wu X M, Yang Y, Cai J, et al. Measurements of ferroelectric MEMS
  %     microphones. Integrated Ferroelectrics, 2005, 69:417-429. (SCI 收录, 检索号
  %     :896KM)
  %   \item 贾泽, 杨轶, 陈兢, 等. 用于压电和电容微麦克风的体硅腐蚀相关研究. 压电与声
  %     光, 2006, 28(1):117-119. (EI 收录, 检索号:06129773469)
  %   \item 伍晓明, 杨轶, 张宁欣, 等. 基于MEMS技术的集成铁电硅微麦克风. 中国集成电路,
  %     2003, 53:59-61.
  % \end{publications}

  \resumeitem{在学期间参加的研究项目} % 有就写,没有就删除
  \begin{enumerate}
  \item 国家自然科学基金项目“基于视觉注意结合生物形态特征的海洋浮游植物显微图像分析”(批准号:61301240)。
  \item 国家自然科学基金项目“基于生物形态特征的中国海常见有害赤潮藻显微图像识别”(批准号:61271406)。
  \item 中央高校基本科研业务费“海洋浮游动物原位探测与分析系统”(批准号:201562023)。
  \end{enumerate}

\end{resume}
