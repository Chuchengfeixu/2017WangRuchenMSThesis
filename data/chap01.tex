\chapter{绪论}
\label{cha:intro}


\section{课题研究的背景及意义}

在我们所生活的地球上,海洋是其表面上最为广阔的水域,占70\%以上。整个海洋环境以及生活在其中的动植物、微生物共同构成了海洋生态系统。在地球上,海洋生态系统为最大的生态系统,其中含有大量的浮游生物。浮游生物是海洋中其他生物的能量来源,因此为构成海洋生态系统必不可少的部分。浮游生物不仅包括浮游植物,还包括浮游动物。在海洋生态系统中,浮游植物是生成者,作为食物链的基本环节可以为其他的生物提供生活所需的能量。此外,浮游植物还会影响全球碳循环和海水中营养物质的浓度。而浮游动物以浮游植物为食,同时还是海洋食物链中高营养级动物的食物。因此,浮游动物作为海洋中生物间能力传递的桥梁,是海洋生态系统中必不可少的部分。综上,浮游动植物的丰富度及分布范围会影响整个海洋生态系统的平衡。

当浮游植物中有害的藻类大量繁殖时,会引发赤潮现象。赤潮会给海洋生态系统带来以下危害:(1)赤潮藻大量聚集会使水体缺氧,鱼类容易窒息死亡;(2)鱼类吞食有毒浮游植物会导致死亡;(3)赤潮发生后会导致水体pH值升高,造成水中生物死亡。因此,浮游植物的大量繁殖将破坏生态平衡,影响渔业的发展,甚至危害人类健康。同时,浮游动物大量繁殖也会对海洋环境造成一定的影响:(1)水域中浮游动物密度较高时会争夺海洋中其他生物的氧气;(2)浮游动物作为海洋食物链中的重要环节,它的丰富程度会影响食物链的平衡;(3)大量浮游动物聚集还会影响水下信号的传播。因此,人们越来越重视对海洋中浮游生物丰富程度的监测,这也是对海洋环境健康程度的一个评价指标。

浮游生物监测是统计在某一时间和空间范围内物种的数量以及丰富度。传统的浮游生物检测先采集样本,然后通过专业人员在显微镜下进行识别和统计,最终计算出该区域中浮游生物的丰富度。然而浮游生物体型小、数量多,人工进行采样、分类和统计不仅需要较高的专业水平,还需要消耗大量的人力、物力和时间。为了提高浮游生物监测的效率,人们研发出了浮游生物图像采集系统,可以方便的采集到水下的浮游生物图像。同时,利用图像处理和模式识别技术设计出浮游生物自动识别系统,对采集的浮游生物显微图像进行自动分类识别,可以实现对浮游生物的自动监测。这两个系统的出现大大提高了对浮游生物丰富度监测的效率,降低了成本。因此,目前人们越来越多的关注于研发性能更好的浮游生物自动分类系统,提高分类的准确率,扩大分类系统的适用范围,使其更好的应用于浮游生物丰富度的监测。



\section{国内外研究现状}
\label{sec:first}

浮游生物的自动监测主要包括浮游生物图像采集和分类识别两个部分,下面主要介绍这两部分的国内外研究现状。

\subsection{浮游生物图像采集技术}

传统的浮游生物采集通常使用网采、瓶采或泵采等方法,这些采集方法存在着一些问题:首先只能在相对较低的时间空间范围内采集样品,而且分析周期很长;其次,拖网容易扰乱浮游生物的分布结构~\cite{孙晓霞2014海洋浮游生物图像观测技术及其应用}。为了克服传统浮游生物采集方法的缺点,在过去的一段时间里浮游生物图像采集系统被广泛的研究与应用。浮游生物图像采集系统通常可以分为两类:实验室成像系统和原位图像采集系统。

实验室成像系统是指在实验室中使用的将浮游生物样本转换为数字图像的仪器设备。浮游动物图像扫描分析系统(ZooScan Integrated System)是一个比较有代表性的实验室成像系统,它由法国人Gorsky.等发明~\cite{grosjean2004enumeration},主要用于对采集到的液体样本中的浮游动物进行成像、检测、识别,该设备由ZooScan、ZooProcess和Plankton Identifier三部分组成。其中ZooScan是扫描成像部分,即图像采集部分,主要负责将采集到的浮游动物样本通过扫描的方式转换成数字图像。ZooProcess和Plankton Identifier主要对ZooScan得到的图像进行处理、测量和自动分类。目前,浮游动物图像扫描分析系统已经广泛应用于浮游动物图像采集和自动分类识别并有较高的效率和分类准确率,在国际上被广泛认可并投入商业化生产~\cite{毕永坤2011基于}。

浮游生物原位图像采集系统可以实时采集水下浮游生物原位图像,保证浮游生物的生存分布结构不被破坏。早在1992年Davis等人研制出了浮游生物视频记录器(Video Plankton Recorder, VPR)~\cite{davis1992video},这是最早的用来采集浮游生物原位图像的系统。后来随着浮游生物原位图像采集系统不断发展,先后出现了水下视频剖面仪(Underwater Video Profiler, UVP)~\cite{davis2004real}、流式细胞仪(Flow Cytometer and Microsocpe, FlowCAM)~\cite{sieracki1998imaging}、灰度图像颗粒探测系统(Shadowed Image Particle Platform and Evaluation Recorder, SIPPER)~\cite{samson2001system}、流式成像技术(Imaging FlowCytobot, IFCB)~\cite{olson2007submersible}等设备。这些设备的出现大大提高了采集浮游生物图像的效率,方便了对水下浮游生物丰富度的监测。

\subsection{浮游生物图像分类技术}
\label{chap1:sample:table} 

传统的浮游生物分类方法是生物学家根据其掌握的专业知识对采集的浮游生物样本进行人工分类。然而海洋中浮游生物种类繁多、形态各异,这使得传统人工分类方法存在以下几个问题:首先,对浮游生物进行分类时需要人员具有较高的专业水平;其次,浮游生物个体小、数量多,人工分类不仅需要大量人力,还会消耗大量时间;第三,人工分类速度较慢,难以实现对浮游生物丰富度的实时监测。因此,人们结合浮游生物图像采集系统研究出了浮游生物自动分类系统。

浮游生物自动分类系统通常采用图像处理和模式识别算法,可以对图像采集设备收集的浮游生物图像进行快速自动分类识别。其中,图像处理是为了获得图像中有用的信息,采用计算机对采集到的数字图像进行去噪、分割、增强并提取特征等操作。

早在20世纪末,硅藻图像数据库已经建立,在对该图像进行自动识别过程中人们结合了图像处理和模式识别方法。在1996年Culverhouse研发了一个对甲藻进行分类的系统,该系统提取图像中细胞的形状和表面特征进行分析,并采用了人工神经网络方法进行分类,实验结果在3种甲藻图像上的总体分类准确率可达72\%。汤晓鸥在1998年\cite{tang1998automatic}提出采用不变矩和傅里叶描述子对浮游生物视频记录器采集的浮游生物图像进行分类,该方法对含有接近2000张图像的浮游生物数据集(包括6类浮游生物)进行分类,得到的分类结果可以达到95\%。后来,在2005~\cite{zhao2005binary}和2006~\cite{tang2006binary}两年中,汤晓鸥还提出采用形状特征对二值浮游生物图像进行分类。Hu等人提出使用灰度共生矩阵描述图像中目标的灰度特征,然后采用支持向量机训练分类器。在2007年Sosik等人~\cite{sosik2007automated}结合多种特征设计了一个浮游生物分类系统,这些特征包括大小、形状、对称性、纹理等,并使用特征选择算法去掉其中的冗余部分,然后针对选择的结果采用支持向量机训练分类器,该分类系统在22类浮游生物图像上的分类准确率达到88\%。之前提到的ZooScan Integrated System~\cite{gorsky2010digital}中的Plankton Identifier是对ZooScan采集的浮游生物图像进行分类和识别的部分,该部分主要根据系统中提取的一系列描述浮游生物的形状、灰度等标量(例如面积、周长、圆形度、灰度对比度、曲率等)来进行分类,该系统在含有20类浮游动物的不平衡数据集上可以得到约78\%的分类准确率。Mosleh等~\cite{mosleh2012preliminary}采用形状和纹理特征对藻类进行描述,然后通过神经网络进行识别。在2015年Ellen等人~\cite{Quantifying2015Ellen}从不同角度对浮游生物分类方法进行分析,研究如何提高浮游生物分类的准确率。

分析国内外的研究现状可以发现,目前的浮游生物自动分类系统已经可以高效、准确的实现对采集的图像进行分类,然而也存在着一些有待改进的方面:(1)现有的分类系统中使用的特征提取方法较为单一,并不能全面的描述图像中浮游生物的形态特征;(2)部分系统在分类过程中使用多种特征描述浮游生物的形态特征,然而在融合不同特征时没有考虑每种特征的贡献比例,并不能充分发挥每种特征的积极作用;(3)由于浮游动物个体相对于浮游植物较大,并且形态相对复杂,因此人们设计的分类系统大多不能同时适用于浮游植物和浮游动物;(4)由于浮游生物种类繁多,国内外目前设计的自动分类系统大多只针对几个特定类别的浮游生物,适用范围较窄。

因此根据以上问题,我们结合多核学习设计了一个浮游生物图像分类系统,该系统从不同角度提取了浮游生物的多种特征,对形态进行全面描述;然后使用多核学习算法将所有特征融合,充分发挥每种特征在分类过程中的积极作用,提高系统的分类性能。同时,该系统具有广泛的适用范围,可以应用于不同的浮游生物数据集(既包括浮游植物也包括浮游动物),具有较好的泛化能力。


\section{课题来源}
\label{sec:complicatedtable}

国家自然科学基金项目“基于视觉注意结合生物形态特征的海洋浮游植物显微图像分析”(批准号:61301240)、国家自然科学基金项目“基于生物形态特征的中国海常见有害赤潮藻显微图像识别”(批准号:61271406)、中央高校基本科研业务费项目“海洋浮游动物原位探测与分析系统”(批准号:201562023)。


\section{论文内容和安排}
\label{sec:tableother}

本文的主要工作内容和安排如下:

第一部分为绪论,该部分主要针对浮游生物分类研究的背景意义、国内外研究现状以及课题来源进行介绍。

第二部分介绍浮游生物图像分类的预备知识,包括浮游生物的基本知识、后续实验使用的数据集、目前应用较广的浮游生物图像分类方法以及评价浮游生物分类系统性能所使用的评价方法。

第三部分根据对浮游生物形态特征的分析以及人们对浮游生物分类识别的过程,选用适合的特征提取方法,既包括简单的几何灰度特征,也包含计算机视觉领域的经典算法,例如局部二值模式、内距离形状上下文等。然后介绍特征选择算法,该方法可以去除提取特征中存在的冗余信息,保留最有效部分,从而降低特征维数,提高分类器的性能。

第四部分首先以支持向量和核函数理论为基础,介绍多核学习的基本思想,以及简单多核学习和非线性多核学习方法。然后,对设计的基于多核学习的浮游生物图像分类系统进行介绍。最后,设计一系列对系统性能进行评价的对比实验,根据对比实验的结果对基于多核学习的浮游生物图像分类系统的性能进行分析。

第五部分对本文研究的浮游生物图像分类工作进行总结,同时针对研究过程中存在的问题进行分析和展望。




