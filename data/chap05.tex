\chapter{总结与展望}

\section{总结}

浮游生物自动分类系统是浮游生物检测的重要部分,它通过图像处理和模式识别方法对采集的浮游生物图像进行自动分类来实现。浮游生物自动分类可以解决人工分类专业水平要求高、费时、费力等问题,提高分类的效率,对浮游生物监测具有重要的研究意义。在本文中,我们从提高浮游生物分类系统的准确率、适用范围等方面出发,以特征提取、多核学习等知识为基础研究基于多核学习的浮游生物分类,主要工作内容如下:
\begin{enumerate}
\item 构建不同的浮游生物数据集,提高分类系统的适用范围。为了验证本文设计的分类系统的泛化能力,我们从网上搜集了不同机构和图像采集设备采集的浮游生物图像,构建三个不同的数据集进行实验,即包含浮游植物数据集,也包含浮游动物数据集。
\item 分析浮游生物的形态特征,选取合适的特征提取方法进行特征描述。人们对浮游生物进行分类时会根据其形状、纹理、大小等特征综合考虑。因此本文在设计浮游生物分类系统时,以人的识别方式为基础,分别提取了浮游生物的以下几类特征:几何灰度特征、纹理特征(例如Gabor滤波器、局部二值模式等)、局部特征(例如形状上下文、尺度不变特征变换等)、粒子测度。从多个角度对浮游生物进行全面描述,提取丰富的浮游生物特征信息,为后续分类做准备。
\item 从提取的特征中为每个数据集选取最优的特征组合,除去其中的冗余部分。采集的丰富的浮游生物特征中会存在不相关或冗余部分,它们会影响分类器的性能。在特征提取后采用特征选择去除其中的冗余部分,减少特征的维数,提高分类的准确率和泛化能力。
\item 分析多核学习理论,选用适合的多核学习方法训练分类器。在本文的分类系统中采用非线性多核学习,针对每种特征选定适合的核函数参数,实现特征的融合并充分发挥每种特征的分类性能。
\item 设计对比实验,建立合理的评价体系,对本文提出的分类系统各部分的性能进行评价。首先根据目前性能较好的浮游生物分类方法设计一个基准分类系统,作为对比评价的基准;然后进行特征对比实验,对分类系统特征提取部分的性能进行评价;最后,用本文设计的浮游生物分类系统进行实验,与之前两个实验对比,对多核学习以及整个分类系统的性能进行评价。
\end{enumerate}

\section{展望}

本文提出的基于多核学习的浮游生物分类系统已经取得了较高的分类准确率,并且用于不同的浮游生物数据集时都有较好的性能,但是对浮游生物图像分类还有几个方面要进行深入研究:
\begin{enumerate}
\item 在多核学习中,每个核函数会对应生成一个核矩阵,核矩阵的数量越多计算量越大,在一定程度上会影响训练时间。因此多核学习相对于单核学习算法的计算速度较慢。如何提高多核学习的计算速度,使该系统可以更好的用于浮游生物实时监测是接下来需要研究的内容之一。
\item 从本文方法在三个数据集上的实验结果可以发现,虽然本文方法对不均衡的数据集(即数据集中每个类别图像数量不等)的分类准确率有一定的提升,然而其分类准确度还低于均衡数据集,因此下一步可以针对数据集不均衡问题进行深入研究。
\end{enumerate}