\thusetup{%封面部分
  %******************************
  % 注意:
  %   1. 配置里面不要出现空行
  %   2. 不需要的配置信息可以删除
  %******************************
  %
  % 中国海洋大学研究生学位论文封面
  % 参考:中国海洋大学研究生学位论文书写格式20130307.doc
  % 为避免出现错误,下面保留[清华大学学位论文模板原有定义无需修改],
  % 请直接跳到后面[中国海洋大学学位论文模板部分请根据自己情况修改]。
  %
%%%%%%%%%%%%%%%%%%%%%%[清华大学学位论文模板原有定义无需修改]%%%%%%%%%%%%%%%%%%%%%%%
  %=====
  % 秘级
  %=====
  secretlevel={秘密},
  secretyear={10},
  %
  %=========
  % 中文信息
  %=========
  ctitle={清华大学学位论文 \LaTeX\ 模板\\使用示例文档 v\version},
  cdegree={工学硕士},
  cdepartment={计算机科学与技术系},
  cmajor={计算机科学与技术},
  cauthor={薛瑞尼},
  csupervisor={郑纬民教授},
  cassosupervisor={陈文光教授}, % 副指导老师
  ccosupervisor={某某某教授}, % 联合指导老师
  % 日期自动使用当前时间,若需指定按如下方式修改:
  % cdate={超新星纪元},
  %
  % 博士后专有部分
  cfirstdiscipline={计算机科学与技术},
  cseconddiscipline={系统结构},
  postdoctordate={2009年7月——2011年7月},
  id={编号}, % 可以留空: id={},
  udc={UDC}, % 可以留空
  catalognumber={分类号}, % 可以留空
  %
  %=========
  % 英文信息
  %=========
  etitle={An Introduction to \LaTeX{} Thesis Template of Tsinghua University v\version},
  % 这块比较复杂,需要分情况讨论:
  % 1. 学术型硕士
  %    edegree:必须为Master of Arts或Master of Science(注意大小写)
  %             “哲学、文学、历史学、法学、教育学、艺术学门类,公共管理学科
  %              填写Master of Arts,其它填写Master of Science”
  %    emajor:“获得一级学科授权的学科填写一级学科名称,其它填写二级学科名称”
  % 2. 专业型硕士
  %    edegree:“填写专业学位英文名称全称”
  %    emajor:“工程硕士填写工程领域,其它专业学位不填写此项”
  % 3. 学术型博士
  %    edegree:Doctor of Philosophy(注意大小写)
  %    emajor:“获得一级学科授权的学科填写一级学科名称,其它填写二级学科名称”
  % 4. 专业型博士
  %    edegree:“填写专业学位英文名称全称”
  %    emajor:不填写此项
  edegree={Doctor of Engineering},
  emajor={Computer Science and Technology},
  eauthor={Xue Ruini},
  esupervisor={Professor Zheng Weimin},
  eassosupervisor={Chen Wenguang},
  % 日期自动生成,若需指定按如下方式修改:
  % edate={December, 2005}
  %
  % 关键词用“英文逗号”分割
  ckeywords={\TeX, \LaTeX, CJK, 模板, 论文},
  ekeywords={\TeX, \LaTeX, CJK, template, thesis}
}

% 定义中英文摘要和关键字
\begin{cabstract}
  论文的摘要是对论文研究内容和成果的高度概括。摘要应对论文所研究的问题及其研究目
  的进行描述,对研究方法和过程进行简单介绍,对研究成果和所得结论进行概括。摘要应
  具有独立性和自明性,其内容应包含与论文全文同等量的主要信息。使读者即使不阅读全
  文,通过摘要就能了解论文的总体内容和主要成果。

  论文摘要的书写应力求精确、简明。切忌写成对论文书写内容进行提要的形式,尤其要避
  免“第 1 章……;第 2 章……;……”这种或类似的陈述方式。

  本文介绍清华大学论文模板 \thuthesis{} 的使用方法。本模板符合学校的本科、硕士、
  博士论文格式要求。

  本文的创新点主要有:
  \begin{itemize}
    \item 用例子来解释模板的使用方法;
    \item 用废话来填充无关紧要的部分;
    \item 一边学习摸索一边编写新代码。
  \end{itemize}

  关键词是为了文献标引工作、用以表示全文主要内容信息的单词或术语。关键词不超过 5
  个,每个关键词中间用分号分隔。(模板作者注:关键词分隔符不用考虑,模板会自动处
  理。英文关键词同理。)
\end{cabstract}

% 如果习惯关键字跟在摘要文字后面,可以用直接命令来设置,如下:
% \ckeywords{\TeX, \LaTeX, CJK, 模板, 论文}

\begin{eabstract}
   An abstract of a dissertation is a summary and extraction of research work
   and contributions. Included in an abstract should be description of research
   topic and research objective, brief introduction to methodology and research
   process, and summarization of conclusion and contributions of the
   research. An abstract should be characterized by independence and clarity and
   carry identical information with the dissertation. It should be such that the
   general idea and major contributions of the dissertation are conveyed without
   reading the dissertation.

   An abstract should be concise and to the point. It is a misunderstanding to
   make an abstract an outline of the dissertation and words ``the first
   chapter'', ``the second chapter'' and the like should be avoided in the
   abstract.

   Key words are terms used in a dissertation for indexing, reflecting core
   information of the dissertation. An abstract may contain a maximum of 5 key
   words, with semi-colons used in between to separate one another.
\end{eabstract}

% \ekeywords{\TeX, \LaTeX, CJK, template, thesis}
%%%%%%%%%%%%%%%%%%%%%%%%%%%%%%%%%%%%%%%%%%%%%%%%%%%%%%%%%%%%%%%%%%%%%%%%%%%%%%%%

%%%%%%%%%%%%%%%%%%[中国海洋大学学位论文模板部分请根据自己情况修改]%%%%%%%%%%%%%%%%%%%
% 中国海洋大学研究生学位论文封面
% 必须填写的内容包括(其他最好不要修改):
%   分类号、密级、UDC
%   论文中文题目、作者中文姓名
%   论文答辩时间
%   封面感谢语
%   论文英文题目
%   中文摘要、中文关键词
%   英文摘要、英文关键词
%
%%%%%[自定义]%%%%%
\newcommand{\fenleihao}{}%分类号
\newcommand{\miji}{}%密级 
                    % 绝密$\bigstar$20年 
                    % 机密$\bigstar$10年
                    % 秘密$\bigstar$5年
\newcommand{\UDC}{}%UDC
\newcommand{\oucctitle}{基于多核学习的浮游生物图像分类研究}%论文中文题目
\ctitle{基于多核学习的浮游生物图像分类研究}%必须修改因为页眉中用到
\cauthor{王如晨}%可以选择修改因为仅在 pdf 文档信息中用到
\cdegree{工学硕士}%可以选择修改因为仅在 pdf 文档信息中用到
\ckeywords{\TeX, \LaTeX, CJK, 模板, 论文}%可以选择修改因为仅在 pdf 文档信息中用到
\newcommand{\ouccauthor}{王如晨}%作者中文姓名
%\newcommand{\ouccsupervisor}{姬光荣教授}%作者导师中文姓名
%\newcommand{\ouccdegree}{博\hspace{1em}士}%作者申请学位级别
%\newcommand{\ouccmajor}{海洋信息探测与处理}%作者专业名称
%\newcommand{\ouccdateday}{\CJKdigits{\the\year}年\CJKnumber{\the\month}月\CJKnumber{\the\day}日}
%\newcommand{\ouccdate}{\CJKdigits{\the\year}年\CJKnumber{\the\month}月}
\newcommand{\oucdatedefense}{         }%论文答辩时间
%\newcommand{\oucdatedegree}{2009年6月}%学位授予时间
\newcommand{\oucgratitude}{谨以此论文献给我的导师和亲人!}%封面感谢语
\newcommand{\oucetitle}{Plankton image classification based on multiple kernel learning}%论文英文题目
%\newcommand{\ouceauthor}{Haiyong Zheng}%作者英文姓名
\newcommand{\oucthesis}{\textsc{OUCThesis}}
%%%%%默认自定义命令%%%%%
% 空下划线定义
\newcommand{\oucblankunderline}[1]{\rule[-2pt]{#1}{.7pt}}
\newcommand{\oucunderline}[2]{\underline{\hskip #1 #2 \hskip#1}}

% 论文封面第一页
%%不需要改动%%
\vspace*{5cm}
{\xiaoer\heiti\oucgratitude

\begin{flushright}
---\hspace*{-2mm}---\hspace*{-2mm}---\hspace*{-2mm}---\hspace*{-2mm}---\hspace*{-2mm}---\hspace*{-2mm}---\hspace*{-2mm}---\hspace*{-2mm}---\hspace*{-2mm}---~\ouccauthor
\end{flushright}
}

\newpage

% 论文封面第二页
%%不需要改动%%
\vspace*{1cm}
\begin{center}
  {\xiaoer\heiti\oucctitle}
\end{center}
\vspace{10.7cm}
{\normalsize\songti
\begin{flushright}
{\renewcommand{\arraystretch}{1.3}
  \begin{tabular}{r@{}l}
    学位论文答辩日期:~ & \oucunderline{2.5cm}{\oucdatedefense} \\
    指导教师签字:~ & \oucblankunderline{5cm} \\
    答辩委员会成员签字:~ & \oucblankunderline{5cm} \\
    ~ & \oucblankunderline{5cm} \\
    ~ & \oucblankunderline{5cm} \\
    ~ & \oucblankunderline{5cm} \\
    ~ & \oucblankunderline{5cm} \\
    ~ & \oucblankunderline{5cm} \\
    ~ & \oucblankunderline{5cm} \\
  \end{tabular}
}
\end{flushright}
}

\newpage

% 论文封面第三页
%%不需要改动%%
\vspace*{1cm}
\begin{center}
  {\xiaosan\heiti 独\hspace{1em}创\hspace{1em}声\hspace{1em}明}
\end{center}
\par{\normalsize\songti\parindent2em
本人声明所呈交的学位论文是本人在导师指导下进行的研究工作及取得的研究成果。据我所知,除了文中特别加以标注和致谢的地方外,论文中不包含其他人已经发表或撰写过的研究成果,也不包含未获得~\oucblankunderline{7cm}(注:如没有其他需要特别声明的,本栏可空)或其他教育机构的学位或证书使用过的材料。与我一同工作的同志对本研究所做的任何贡献均已在论文中作了明确的说明并表示谢意。
}
\vskip1.5cm
\begin{flushright}{\normalsize\songti
  学位论文作者签名:\hskip2cm 签字日期:\hskip1cm 年 \hskip0.7cm 月\hskip0.7cm 日}
\end{flushright}
\vskip.5cm
{\setlength{\unitlength}{0.1\textwidth}
  \begin{picture}(10, 0.1)
    \multiput(0,0)(0.2, 0){50}{\rule{0.15\unitlength}{.5pt}}
  \end{picture}}
\vskip1cm
\begin{center}
  {\xiaosan\heiti 学位论文版权使用授权书}
\end{center}
\par{\normalsize\songti\parindent2em
本学位论文作者完全了解学校有关保留、使用学位论文的规定,并同意以下事项:
\begin{enumerate}
\item 学校有权保留并向国家有关部门或机构送交论文的复印件和磁盘,允许论文被查阅和借阅。
\item 学校可以将学位论文的全部或部分内容编入有关数据库进行检索,可以采用影印、缩印或扫描等复制手段保存、汇编学位论文。同时授权清华大学“中国学术期刊(光盘版)电子杂志社”用于出版和编入CNKI《中国知识资源总库》,授权中国科学技术信息研究所将本学位论文收录到《中国学位论文全文数据库》。
\end{enumerate}
(保密的学位论文在解密后适用本授权书)
}
\vskip1.5cm
{\parindent0pt\normalsize\songti
学位论文作者签名:\hskip4.2cm\relax%
导师签字:\relax\hspace*{1.2cm}\\
签字日期:\hskip1cm 年\hskip0.7cm 月\hskip0.7cm 日\relax\hfill%
签字日期:\hskip1cm 年\hskip0.7cm 月\hskip0.7cm 日\relax\hspace*{1.2cm}}

\newpage

\pagestyle{plain}
\clearpage\pagenumbering{roman}

% 中文摘要
%%[需要填写:中文摘要、中文关键词]%%
\begin{center}
  {\sanhao[1.5]\heiti\oucctitle\\\vskip7pt 摘\hspace{1em}要}
\end{center}
{\normalsize\songti

  \indent
  浮游生物是海洋中生物的基本能量来源,构成了海洋食物链的基础,因此其丰富程度会影响海洋生态系统的平衡。并且,浮游生物对环境的改变较为敏感,专家们也可以利用其这一特点研究环境的变化。此外,大量有害浮游生物繁殖还会造成污染。因此,监测海洋浮游生物的分布和丰富度对海洋生态系统研究以及海洋环境保护等工作具有重要意义。

  早期对浮游生物的监测是通过人工采集浮游生物样本,并由专业人士借助显微镜等设备进行分类统计实现。这个过程不仅耗时耗力,而且专业要求高,因此导致浮游生物监测效率低下。为了解决这一问题,人们先后研制了浮游生物图像采集系统和自动识别系统,这两个系统可以高效的采集并识别大量浮游生物图像,他们的出现不仅提高浮游生物监测的效率,也降低了成本和专业需求。然而,现有浮游生物图像分类系统的泛化能力和分类性能有待提高。因此,本文以提高浮游生物图像分类系统整体性能为目标,研究并提出了基于多核学习的浮游生物图像分类系统,以提高浮游生物分类的准确率和适用范围,使其更好的应用于浮游生物监测中,主要工作如下:

  \begin{itemize}
    \item 构建不同的浮游生物数据集。为了降低所设计图像分类系统的数据集偏见,提高整个系统的泛化能力,我们搜集了不同机构和设备采集的浮游生物图像,用其构建不同的数据集进行实验,这些数据集中既包含浮游动物数据集,也包含浮游植物数据集。
    \item 研究从多角度对浮游生物的形态特征进行分析,提出结合多种有效特征的浮游生物描述方法。首先,根据浮游生物的形态,选用适合的统计方法提取几何灰度等特征;此外,将目标检测与识别领域的经典特征提取方法应用于浮游生物描述,例如局部二值模式、梯度方向直方图、内距离形状上下文等算法;然后,针对提取特征中的冗余部分,采用特征选择算法将其去除,为每个数据集保留最优的特征组合。
    \item 提出基于多核学习的浮游生物图像分类方法。对于提取的不同种类的浮游生物特征,分别为其设计适合的核函数,利用多核学习算法进行融合,从而充分发挥每种特征在分类中的积极作用。
    \item 设计对比实验,构建合理的评价体系,对提出的基于多核学习的浮游生物图像分类系统的性能进行评价。首先,根据目前应用较好的浮游生物图像分类方法设计基准实验,作为评价分类器性能的基准;然后在基准实验的基础上设计特征对比实验,对特征提取部分的性能进行评价;最后,采用本文设计的分类系统进行实验,与之前实验结果进行对比评价。
  \end{itemize}

  实验结果表明,本文提出的基于多核学习的浮游生物图像分类系统具有较好的分类性能和泛化能力,在不同浮游生物数据集上都取得了较好的分类结果。
}
\vskip12bp
{\xiaosi\heiti\noindent
关键词:\hskip1em 浮游生物,多核学习,特征提取,核函数,特征融合}

\newpage

% 英文摘要
%%[需要填写:英文摘要、英文关键词]%%
\begin{center}
  {\sanhao[1.5]\heiti\oucetitle\\\vskip7pt Abstract}
\end{center}
{\normalsize\songti

   Plankton is the main source of food for organisms in the ocean and forms the base of marine food chain. So the abundance of it will influence the ocean ecological balance. In addition, plankton is very sensitive to environment changes, thus it can be used to study the changing environment. And harmful plankton bloom pollutes the marine environment. Therefore, the study of plankton abundance and distribution is important, in order to understand environment change and protect marine ecosystems. 

   In the early days, researchers monitor the distribution and abundance of plankton with manually collecting plankton samples and classification by experts. The aforementioned process is so laborious and time consuming that hinders the plankton monitoring. To solve the problem, several imaging devices have been developed for collecting plankton images, and automatic classification system of plankton images also have been developed. The invention of imaging devices and classification system improves the efficiency of plankton monitoring. Meanwhile, the cost and specialized demand are reduced. At present, the generalization ability and classification performance wait for enhancing. This thesis studies the plankton image classification based on multiple kernel learning to improve the performance of classification system. So the system can be better used to monitor plankton. The main researches are as follows:

   \begin{itemize}
   \item Different plankton datasets are established for experiments. In order to reduce dataset bias and improve generalization ability, the plankton images gathered by different organizations and imaging devices are collected to build different datasets for experimental purposes. These datasets involve both phytoplankton dataset and zooplankton dataset.
   \item From various angles to analyze the characteristic of plankton, feasible and valid features can be combined and applied to decribe the characteristic of plankton. During the process of research, firstly, according to the characteristic of plankton, statistical methods are used to extract geometric and gray features. In addition, classical algorithms in object detection are applied to describe characteristic of plankton, such as Local Binary Pattern, Histogram of Oriented Gradients, Inner-Distance Shape Context and so on. Then, the redundant information of extracted feature is reduced by feature selection to choose the best features for each dataset.
   \item The system based on multiple kernel learning is proposed for plankton classification. In this process, predefined kernels are chose for each type of feature, and the optimal linear or non-linear combination of all kernels is learned with multiple kernel learning. So that features can give full play in classification.
   \item Contrast experiments are designed to establish system to evaluate the performance of plankton classification system. Firstly, according to the successful methods of plankton classification method at present, a baseline experiment is designed as the standard of contrast. Then, the second experiment is designed based on the baseline experiment to evaluate the property of feature extraction methods. Finally, we carry out the plankton classification system based on multiple kernel learning, and evaluate the performance of it by comparing with previous experiments.
   \end{itemize}

   Experimental results show that the plankton classification system based on multiple kernel learning has better classification performance and generalization ability. It is feasible and effective in plankton monitoring.
}
\vskip12bp
{\xiaosi\heiti\noindent 
\textbf{Keywords:\enskip plankton, multiple kernel learning, feature extraction, kernel function, feature fusion}}
%%%%%%%%%%%%%%%%%%%%%%%%%%%%%%%%%%%%%%%%%%%%%%%%%%%%%%%%%%%%%%%%%%%%%%%%%%%%%%%%